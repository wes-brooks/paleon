\documentclass[authoryear, review, 11pt]{elsarticle}

\setlength{\textwidth}{6.5in}
%\setlength{\textheight}{9in}
\setlength{\topmargin}{0in}
\setlength{\oddsidemargin}{0in}
\setlength{\evensidemargin}{0in}

\usepackage{amsmath}
\usepackage{amsthm}
\usepackage{amssymb}
\usepackage{mathabx}
\usepackage{bm}
\usepackage{multirow}

%\geometry{landscape}                % Activate for for rotated page geometry
\usepackage[parfill]{parskip}    % Activate to begin paragraphs with an empty line rather than an indent
\usepackage{graphicx}
\usepackage{epstopdf}
\usepackage{natbib}
\usepackage{verbatim}

\usepackage{endfloat}

\usepackage{relsize}
\usepackage{caption}
\usepackage{subcaption}
%\usepackage{fullpage}
\usepackage{booktabs}

\newcommand{\ig}{\includegraphics}

\DeclareGraphicsRule{.tif}{png}{.png}{`convert #1 `dirname #1`/`basename #1 .tif`.png}
\DeclareMathOperator*{\argmin}{\arg\!\min}
\DeclareMathOperator*{\argmax}{\arg\!\max}
\DeclareMathOperator*{\bw}{\mbox{bw}}
\DeclareMathOperator*{\df}{\mbox{df}}
\newcommand{\vect}[1]{\bm{#1}}
\newcommand{\E}{\mathop{\mathbb E}}


% title info
\title{Spatial smoothing of zero-inflated abundance data}
%\subtitle{A researcher's perspective}
\author{Wesley Brooks}




\begin{document}
\maketitle

\section{Introduction}
Data for this project are estimates of per-taxon biomass based on the Public Land Survey. The data are biomass estimates for each grid cell in each taxon. Each grid cell is observed at about 70 corner points.


Our objective is a  model of the biomass of each species in each grid cell, where the only parameters used to calibrate the models are location and species composition. It is important to come up with an estimated probability that a given cell will have no biomass of a given species, and also to model the variance of the biomass estimate.\\


\section{Data}





\section{Models}
The model for biomass is a Tweedie model, which can be thought of as a Poisson-Gamma mixture. That is, letting the response be $Y$,
\begin{align*}
    Y &= \sum\limits_{i=1}^N Z_i \\
    N &\sim \text{Poisson}(\lambda) \\
    Z_i &\sim \text{Gamma}(\alpha, \tau)
\end{align*}

 $\theta \in (1,2)$:
\begin{align*}
    \lambda &= \phi^{-1} \frac{\mu^{2-\theta}}{2-\theta} \\
    \alpha &= \frac{2-\theta}{\theta-1} \\
    \tau &= \phi (\theta - 1) \mu^{\theta - 1}
\end{align*}

And, as usual for a generalized linear model, we define a link function, $g(\dot)$, that converts the fitted values $\hat{y}$ to the linear predictor $\eta$. In this case, the log link is used.
\begin{align*}
    \eta(s) &= f\left(\bm{x}(s)\right) = g\left(\mu(s)\right)\\
    \mu(s) &= g^{-1}\left(\eta(s)\right)
\end{align*}

Where $g(\cdot)$ is the link function, $\bm{x}(s)$ is the vector of covariates ($x$ and $y$ coordinates) at location $s$.


\section{Computation}
The GAM was fit using 500 knots for Wisconsin. At 



\section{Results}
\begin{figure}
\begin{centering}
\ig[width=0.7\textwidth]{../../figures/biomass/hardwood-histogram}
\caption{Distribution of the total hardwood biomass for Wisconsin. The vertical line represents the sum of the Wisconsin hardwood biomass observations.\label{hardwood-hist}}
\end{centering}
\end{figure}

\begin{figure}
\begin{centering}
\ig[width=0.7\textwidth]{../../figures/biomass/softwood-histogram}
\caption{Distribution of the total softwood biomass for Wisconsin. The vertical line represents the sum of the Wisconsin softwood biomass observations.\label{softwood-hist}}
\end{centering}
\end{figure}


\end{document}
