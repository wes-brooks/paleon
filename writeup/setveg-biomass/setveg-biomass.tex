% small.tex
\documentclass{beamer}
\usetheme{Boadilla}
\setbeamertemplate{blocks}[rounded][shadow=false] 

\usepackage{subfig}
\usepackage{multirow}
\usepackage{amsmath}
\usepackage{mathtools}
\usepackage{listings}
\usepackage{color}
 
\definecolor{dkgreen}{rgb}{0,0.6,0}
\definecolor{gray}{rgb}{0.5,0.5,0.5}
\definecolor{mauve}{rgb}{0.58,0,0.82}

\lstset{ %
  language=Python,                % the language of the code
  basicstyle=\footnotesize,           % the size of the fonts that are used for the code
  %numbers=left,                   % where to put the line-numbers
  %numberstyle=\tiny\color{gray},  % the style that is used for the line-numbers
  %stepnumber=2,                   % the step between two line-numbers. If it's 1, each line 
                                  % will be numbered
  %numbersep=5pt,                  % how far the line-numbers are from the code
  %backgroundcolor=\color{white},      % choose the background color. You must add \usepackage{color}
  showspaces=false,               % show spaces adding particular underscores
  showstringspaces=false,         % underline spaces within strings
  showtabs=false,                 % show tabs within strings adding particular underscores
  %frame=single,                   % adds a frame around the code
  rulecolor=\color{black},        % if not set, the frame-color may be changed on line-breaks within not-black text (e.g. commens (green here))
  tabsize=2,                      % sets default tabsize to 2 spaces
  captionpos=b,                   % sets the caption-position to bottom
  breaklines=true,                % sets automatic line breaking
  breakatwhitespace=false,        % sets if automatic breaks should only happen at whitespace
  title=\lstname,                   % show the filename of files included with \lstinputlisting;
                                  % also try caption instead of title
  keywordstyle=\color{blue},          % keyword style
  commentstyle=\color{dkgreen},       % comment style
  stringstyle=\color{mauve},         % string literal style
  escapeinside={\%*}{*)},            % if you want to add a comment within your code
  morekeywords={dynamic, string}               % if you want to add more keywords to the set
}


\AtBeginSection[]
{
  \begin{frame}
    \frametitle{Table of Contents}
    \tableofcontents[currentsection]
  \end{frame}
}


%About me
\author{Wesley Brooks} 
\title{Modeling PalEON biomass}
%\subtitle{Wesley Brooks} 
\institute{UW-Madison} 

\begin{document}

%Title slide
\begin{frame}
\titlepage
\end{frame}


%Table of contents
\begin{frame}{Outline}
  \tableofcontents
\end{frame}


%Goal
\begin{frame}{Goal}
  \begin{itemize}
    \item Produce a model of per-species biomass at time of settlement
  \end{itemize}
\end{frame}

\section{Data}
\subsection{Overview of the data}

% Data
\begin{frame}{Data}
  \begin{itemize}
    \item Computed from settlement-era survey
    \item Working with composition, biomass, and stem density
  \end{itemize}
\end{frame}


%
\begin{frame}{}
\begin{center}
  %\includegraphics[width=0.8\textwidth]{figures/lh-results}
\end{center}
\end{frame}


\subsection{Models}


%Model types
\begin{frame}{Models}
  There are two divisions for modeling biomass data:
  \begin{itemize}
    \item One-stage vs. two-stage
    \item Smoothing splines vs. GMRF
   \end{itemize}
\end{frame}


  %Models
\begin{frame}[fragile]{Two-stage models}
  \begin{itemize}
    \item First stage: zero/non-zero
    \begin{itemize}
      \item Logistic regression
      \item $Z \sim \text{Bernoulli}(\gamma)$
      \item $\text{logit}(\gamma) = f(x,y,p_k)$
    \end{itemize}
    \item Second stage: distribution of positive biomass
    \begin{itemize}
      \item $Y|Z=1 \sim \text{Gamma}(\alpha, \beta)$
      \item $\text{E}\left(Y|Z=1\right) = \mu = \alpha \beta = f(x,y,p_k)$
    \end{itemize}
  \end{itemize}
%\begin{lstlisting}
%for (cycle in 1:40):
%    count the gene copies
%    use PCR to produce a new generation
%\end{lstlisting}
\end{frame}


  %Models
\begin{frame}[fragile]{Tweedie model}
  The Tweedie model is a Gamma-Poisson mixture.\\
  How to visualize a Tweedie random variable:
  \begin{itemize}
    \item Draw $N \sim \text{Poisson}(\lambda)$
    \item Now make $N$ iid draws: $V_{\ell} \sim \text{Gamma}(\alpha, \beta)$
    \item $Y = \sum\limits_{\ell=1}^N  V_{\ell}$
  \end{itemize}
\end{frame}


\section{Methodological details}
\subsection{}


%Sources of randomness
\begin{frame}{}

\end{frame}


\subsection{Branching process details}


%qPCR is a branching process
\begin{frame}{qPCR as a branching process}
Note:\\

\begin{align*}
  E[N_{n}] &= E[E(N_{n}|N_{n-1})] = E[(1+p)N_{n-1}] \\
  &= \cdots = (1+p)^n \times E(N_0)
\end{align*}
\vspace{3mm}
\begin{itemize}
  \item So $W_n = \frac{N_n}{(1+p)^n}$ is a positive martingale
  \vspace{3mm}
  \item Thus, $W_n \to W$ almost surely for some $W$
  \vspace{3mm}
  \item $E[W] = m_a$
\end{itemize}
\end{frame}


%qPCR as a branching process
\begin{frame}{qPCR as a branching process}
Consider an idealized reaction experiment:
  \begin{itemize}
    \item If we knew p and could let the number of reaction cycles $n \to \infty$:
    \begin{itemize}
    	\vspace{3mm}
    	\item $W_i = \lim_{n \to \infty} \frac{N_{i,n}}{(1+p)^n} $
	\vspace{2mm}
	\item $W_1, W_2, \ldots, W_r \stackrel{\text{iid}}{\sim} W$
	\vspace{4mm}
	\item So $\frac{1}{r} \Sigma_{i=1}^r W_i \stackrel{\text{a.s.}}{\to} E[W] = m_a$
    \end{itemize}
    \item But p is unknown and we can only observe $\approx$ 15 reaction cycles, so we need some other estimator.
  \end{itemize}
\end{frame}


%qPCR as a branching process
\begin{frame}{Estimating p}
  \begin{itemize}
    \item Since $p$ is unknown, we estimate it with $\hat{p}$ via weighted least squares:
\[
\begin{pmatrix}
N_n\\
N_{n-1}\\
\vdots\\
N_1\\
\end{pmatrix}
 - 
\begin{pmatrix}
N_{n-1}\\
N_{n-2}\\
\vdots\\
N_0\\
\end{pmatrix}
 = p \times
\begin{pmatrix}
N_{n-1}\\
N_{n-2}\\
\vdots\\
N_0\\
\end{pmatrix}
 + \epsilon
\]
\item Where $\epsilon_j \stackrel{\text{approx.}}{\sim} \text{Normal}(0, p(1-p) N_{j-1})$
\vspace{2mm}
\item With weights $W_j = (N_{j-1})^{-1}$ the resulting estimator is:
\[
\hat{p} = \frac{\Sigma_{i=1}^n (N_i - N_{i-1})}{\Sigma_{i=1}^n N_i}
\]
  \end{itemize}
\end{frame}


%qPCR as a branching process
\begin{frame}{Making the most of a finite sample}
Reminder: our idealized estimator was $W(n) = \frac{N_n}{(1+p)^n} $
\vspace{3mm}
  \begin{itemize}
    \item $W$ uses only the final observation ($N_n$)
    \vspace{3mm}
    \item More efficient: use the sum $Y_n = \Sigma_{i=1}^n N_i$
    \vspace{3mm}
    \item By the Toeplitz Lemma, $\frac{Y_n}{(1+p)^n} \stackrel{\text{a.s.}}{\to} \frac{1+p}{p}W \Rightarrow \frac{pY_n}{(1+p)^{n+1}} \stackrel{\text{a.s.}}{\to} W$
    \vspace{3mm}
    \item Plug in $\hat{p}$ and the limit still holds.
  \end{itemize}
\end{frame}


\subsection{Estimating gene copies from qPCR}


%qPCR as a branching process
\begin{frame}{Strategy for quantitation}
  \begin{itemize}
    \item Collect data on $r$ independent reactions
    \vspace{3mm}
    \item For reaction $i$ ($i=1, 2, \dots, r$), compute the statistic $M_i = \frac{\hat{p}_i Y_{n_i}}{(1+\hat{p}_i)^{n_i+1}}$
    \vspace{3mm}
    \item Average $M_1, M_2, \dots, M_r$ to get $\bar{M}$
    \vspace{3mm}
    \item $\sqrt{r}(\bar{M} - m_a) \stackrel{d}{\to} \text{Normal}(0,\sigma^2_L)$
    \vspace{3mm}
    \item Where $\sigma^2_L = \sigma^2_a + m_a E[\frac{1-p}{1+p}]$
  \end{itemize}
\end{frame}


\subsection{Variance of the estimator}


%qPCR as a branching process
\begin{frame}{Variance of the estimator}
\begin{align*}
\sigma^2_L &= \text{var}[\frac{N_n}{(1+p)^n}] = E(\text{var}[\frac{N_n}{(1+p)^n}|p]) + \text{var}(E[\frac{N_n}{(1+p)^n}|p])\\
&= E(\text{var}[\frac{N_n}{(1+p)^n}|p])  + \text{var}(m_a)\\
&= E(\text{var}[\frac{N_n}{(1+p)^n}|p])
\end{align*}
\end{frame}


%qPCR as a branching process
\begin{frame}{Variance of the estimator}
\begin{align*}
\text{var}[\frac{N_n}{(1+p)^n}|p] &= \frac{1}{(1+p)^{2n}}\text{var}[N_n|p] \\
&= \frac{1}{(1+p)^{2n}} ( E(\text{var}[N_n|N_{n-1},p]|p) + \text{var}(E[N_n|N_{n-1},p]|p) ) \\
&= \frac{1}{(1+p)^{2n}} ( E[N_{n-1}p(1-p)|p] + \text{var}((1+p)N_{n-1}|p) ) \\
&= \frac{1}{(1+p)^{2n}} ( m_a (1+p)^{n-1}p(1-p) + (1+p)^2\text{var}[N_{n-1}|p] ) \\
&= \frac{m_a p (1-p)}{(1+p)^{n+1}}  + \frac{\text{var}[N_{n-1}|p]}{(1+p)^{2n-2}} \\
&= \dots \\
&= \frac{m_a p (1-p)}{(1+p)^{n+1}}  + \frac{m_a p (1-p)}{(1+p)^{n}} + \dots + \frac{m_a p (1-p)}{(1+p)^{2}} \\ &+ \frac{\text{var}[N_{0}|p]}{(1+p)^{2n-2n}} \\
&= \frac{m_a p (1-p)}{(1+p)^{n+1}}  + \frac{m_a p (1-p)}{(1+p)^{n}} + \dots + \frac{m_a p (1-p)}{(1+p)^{2}} \\ 
& + \frac{\text{var}[N_{0}|p]}{(1+p)^{2n-2n}} \\
\end{align*}
\end{frame}


%qPCR as a branching process
\begin{frame}{Variance of the estimator}
\begin{align*}
\text{var}[\frac{N_n}{(1+p)^n}|p] &= \frac{m_a p (1-p)}{(1+p)^{2}} \Sigma_{k=0}^{n-1}\frac{1}{(1+p)^k} + \sigma^2_a \\ 
&\to m_a  \frac{1-p}{1+p} + \sigma^2_a \\
\end{align*}
So:
\begin{align*}
\text{var}[\frac{N_j}{(1+p)^n}] &= E(\text{var}[\frac{N_n}{(1+p)^n}|p])\\
&\to E(m_a \frac{1-p}{1+p} + \sigma^2_a) \\
&= m_a E(\frac{1-p}{1+p}) + \sigma^2_a = \sigma^2_L
\end{align*}
\end{frame}


\section{Analysis of experimental data}
\subsection{Luteinizing hormone}


%Description of experimental data (luteinizing hormone)
\begin{frame}{Experimental data - luteinizing hormone} 
\begin{itemize}
  \item The goal with the experimental data was \emph{relative} quantitation
  \begin{itemize}
    \item Estimate ratio of gene expression between conditions C and T
  \end{itemize}
  \item The sample was divided into two parts
  \item One part was diluted to one-third the original concentration
  \item Sixteen reactions were run under each condition (diluted, normal)
\end{itemize}
\end{frame} 


%results of experiment (luteinizing hormone)
\begin{frame}{Experimental data - luteinizing hormone} 

\end{frame} 


\end{document}
