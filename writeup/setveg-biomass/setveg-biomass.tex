% small.tex
\documentclass{beamer}
\usetheme{Boadilla}
\setbeamertemplate{blocks}[rounded][shadow=false] 

\usepackage{subfig}
\usepackage{multirow}
\usepackage{amsmath}
\usepackage{mathtools}
\usepackage{listings}
\usepackage{color}
\usepackage{multirow}
 
\definecolor{dkgreen}{rgb}{0,0.6,0}
\definecolor{gray}{rgb}{0.5,0.5,0.5}
\definecolor{mauve}{rgb}{0.58,0,0.82}

\lstset{ %
  language=Python,                % the language of the code
  basicstyle=\footnotesize,           % the size of the fonts that are used for the code
  %numbers=left,                   % where to put the line-numbers
  %numberstyle=\tiny\color{gray},  % the style that is used for the line-numbers
  %stepnumber=2,                   % the step between two line-numbers. If it's 1, each line 
                                  % will be numbered
  %numbersep=5pt,                  % how far the line-numbers are from the code
  %backgroundcolor=\color{white},      % choose the background color. You must add \usepackage{color}
  showspaces=false,               % show spaces adding particular underscores
  showstringspaces=false,         % underline spaces within strings
  showtabs=false,                 % show tabs within strings adding particular underscores
  %frame=single,                   % adds a frame around the code
  rulecolor=\color{black},        % if not set, the frame-color may be changed on line-breaks within not-black text (e.g. commens (green here))
  tabsize=2,                      % sets default tabsize to 2 spaces
  captionpos=b,                   % sets the caption-position to bottom
  breaklines=true,                % sets automatic line breaking
  breakatwhitespace=false,        % sets if automatic breaks should only happen at whitespace
  title=\lstname,                   % show the filename of files included with \lstinputlisting;
                                  % also try caption instead of title
  keywordstyle=\color{blue},          % keyword style
  commentstyle=\color{dkgreen},       % comment style
  stringstyle=\color{mauve},         % string literal style
  escapeinside={\%*}{*)},            % if you want to add a comment within your code
  morekeywords={dynamic, string}               % if you want to add more keywords to the set
}


\AtBeginSection[]
{
  \begin{frame}
    \frametitle{Table of Contents}
    \tableofcontents[currentsection]
  \end{frame}
}


%About me
\author{Wesley Brooks} 
\title{Modeling PalEON biomass}
%\subtitle{Wesley Brooks} 
\institute{UW-Madison} 

\begin{document}

%Title slide
\begin{frame}
  \titlepage
\end{frame}


%Table of contents
\begin{frame}{Outline}
  \tableofcontents
\end{frame}


%Goal
\begin{frame}{Goal}
  \begin{itemize}
    \item Produce a model of per-species biomass at time of settlement
  \end{itemize}
\end{frame}

\section{Data}
\subsection{Overview of the data}

% Data
\begin{frame}{Data}
  \begin{itemize}
    \item Computed from settlement-era survey
    \item Working with composition, biomass, and stem density
  \end{itemize}
\end{frame}


%
\begin{frame}{}
\begin{center}
  %\includegraphics[width=0.8\textwidth]{figures/lh-results}
\end{center}
\end{frame}


\section{Modeling biomass}

%Modeling table
\begin{frame}{Models}
  There are two divisions for modeling biomass data:
  \begin{center}
    \begin{tabular}{cc|r|c|c|c|}
    & \multicolumn{2}{c}{} & \multicolumn{3}{c}{Estimation method}\\
    & \multicolumn{2}{c}{} & \multicolumn{1}{c}{\rotatebox{60}{Splines}} & \multicolumn{1}{c}{\rotatebox{60}{GMRF}} & \multicolumn{1}{c}{\rotatebox{60}{GLM}}\\
    \cline{3-6}
    \multirow{3}{*}{\rotatebox{90}{Model}}  & \multirow{2}{*}{One-stage} & Zero-inflated gamma &  &  & \\
    \cline{4-6}
    & & Tweedie &  &  & \\
    \cline{3-6}
    & Two-stage & Bernoulli-Gamma & & & \\
    \cline{3-6}
    \end{tabular}
   \end{center}
\end{frame}

\subsection{Models}

  %Model types
\begin{frame}[fragile]{Two-stage models}
  \begin{itemize}
    \item First stage: zero/non-zero
    \begin{itemize}
      \item Logistic regression
      \item $Z \sim \text{Bernoulli}(\gamma)$
      \item $\text{logit}(\gamma) = f(x,y,p_k)$
    \end{itemize}
    \item Second stage: distribution of positive biomass
    \begin{itemize}
      \item $Y|Z=1 \sim \text{Gamma}(\alpha, \beta)$
      \item $\text{E}\left(Y|Z=1\right) = \mu = \alpha \beta = f(x,y,p_k)$
    \end{itemize}
  \end{itemize}
%\begin{lstlisting}
%for (cycle in 1:40):
%    count the gene copies
%    use PCR to produce a new generation
%\end{lstlisting}
\end{frame}


  %Models
\begin{frame}[fragile]{Tweedie model}
  The Tweedie model is a Gamma-Poisson mixture.\\
  How to visualize a Tweedie random variable:
  \begin{itemize}
    \item Draw $N \sim \text{Poisson}(\lambda)$
    \item Now make $N$ iid draws: $V_{\ell} \sim \text{Gamma}(\alpha, \beta)$
    \item $Y = \sum\limits_{\ell=1}^N  V_{\ell}$
  \end{itemize}
\end{frame}


\subsection{Fitting}

\begin{frame}[fragile]{GLMs, Independent observations}
  Based on composition fraction, ignoring spatial location
\end{frame}

\begin{frame}[fragile]{GMRF}
\end{frame}

\begin{frame}[fragile]{Splines}
\end{frame}

\section{Methodological details}
\subsection{}


%Sources of randomness
\begin{frame}{}

\end{frame}


\subsection{Branching process details}

%Sources of randomness
\begin{frame}{}

\end{frame}


\end{document}
