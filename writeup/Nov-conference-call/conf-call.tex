\documentclass[12pt,t]{beamer}
\usepackage{graphicx}
\usepackage{bm}
\usepackage{verbatim}
\usepackage{xcolor}
\usepackage{multirow}
\setbeameroption{hide notes}
\setbeamertemplate{note page}[plain]

% get rid of junk
\usetheme{default}
\beamertemplatenavigationsymbolsempty
\hypersetup{pdfpagemode=UseNone} % don't show bookmarks on initial view

% tables
\usepackage{tabularx}
\newcolumntype{Y}{>{\centering\arraybackslash}X}

% font
\usefonttheme{professionalfonts}
\usefonttheme{serif}
\usepackage{fontspec}
\setmainfont{Helvetica Neue}
\setbeamerfont{note page}{family*=pplx,size=\footnotesize} % Palatino for notes
\setbeamerfont{frametitle}{size=\large}

% named colors
\definecolor{offwhite}{RGB}{249,242,215}
\definecolor{foreground}{RGB}{25,25,25}
\definecolor{background}{RGB}{250,250,245}
\definecolor{title}{RGB}{20,20,20}
\definecolor{gray}{RGB}{80,80,80}
\definecolor{subtitle}{RGB}{20,20,20}
\definecolor{hilight}{RGB}{255,127,0}
\definecolor{vhilight}{RGB}{255,111,207}
\definecolor{lolight}{RGB}{155,155,155}
%\definecolor{green}{RGB}{125,250,125}

% use those colors
\setbeamercolor{titlelike}{fg=title}
\setbeamercolor{subtitle}{fg=subtitle}
\setbeamercolor{institute}{fg=gray}
\setbeamercolor{normal text}{fg=foreground,bg=background}
\setbeamercolor{item}{fg=foreground} % color of bullets
\setbeamercolor{subitem}{fg=gray}
\setbeamercolor{itemize/enumerate subbody}{fg=gray}
\setbeamertemplate{itemize subitem}{{\textendash}}
\setbeamerfont{itemize/enumerate subbody}{size=\footnotesize}
\setbeamerfont{itemize/enumerate subitem}{size=\footnotesize}

% page number
\setbeamertemplate{footline}{%
    \raisebox{5pt}{\makebox[\paperwidth]{\hfill\makebox[20pt]{\color{gray}
          \scriptsize\insertframenumber}}}\hspace*{5pt}}

% add a bit of space at the top of the notes page
\addtobeamertemplate{note page}{\setlength{\parskip}{12pt}}

% a few macros
\newcommand{\bi}{\begin{itemize}}
\newcommand{\ei}{\end{itemize}}
\newcommand{\ig}{\includegraphics}
\newcommand{\subt}[1]{{\footnotesize \color{subtitle} {#1}}}
\DeclareGraphicsRule{.tif}{png}{.png}{`convert #1 `dirname #1`/`basename #1 .tif`.png}
\DeclareMathOperator*{\argmin}{\arg\!\min}
\DeclareMathOperator*{\argmax}{\arg\!\max}
\DeclareMathOperator*{\bw}{\mbox{bw}}
\DeclareMathOperator*{\df}{\mbox{df}}
\newcommand{\vect}[1]{\bm{#1}}
\newcommand{\E}{\mathop{\mathbb E}}

% title info
\title{Spatial smoothing of zero-inflated abundance data}
%\subtitle{A researcher's perspective}
\author{\href{http://www.somesquares.org}{Wesley Brooks}}
\institute{\href{http://www.stat.wisc.edu}{Department of Statistics} \\[2pt] \href{http://www.wisc.edu}{University of Wisconsin{\textendash}Madison}}
\date{\href{http://www.somesquares.org}{\tt \scriptsize . }}

%bibliography stuff
\usepackage[autostyle]{csquotes}
\usepackage[backend=biber,style=authoryear,natbib=true]{biblatex}
\addbibresource{../../references/paleon.bib}



\setbeamerfont{section title}{parent=title,size=\large}
\setbeamercolor{section title}{parent=titlelike}
\defbeamertemplate*{mod section page}{default}[1][]
{
  \centering
    \vspace{3cm}
    \begin{beamercolorbox}[sep=8pt,center,#1]{section title}
      \usebeamerfont{section title}\insertsection\par
    \end{beamercolorbox}
}
\newcommand*{\modsectionpage}{\usebeamertemplate*{mod section page}}

\AtBeginSection{\frame{\modsectionpage}}


\begin{document}

% title slide
{
  \setbeamertemplate{footline}{} % no page number here
  \frame{
    \titlepage
    \note{These slides were prepared for a practice version of my preliminary exam to advance to Ph.D candidacy in statistics at the University of Wisconsin{\textendash}Madison.}
  }
}




\begin{frame}{Motivation}
\subt{A look at the data}

The data is the total tree biomass per grid cell from the Public Land Survey of the upper midwest.

\begin{itemize}
    \item Observations are at corner points (~90 per grid cell)
    \item Want to know the actual biomass on the grid cell from these local samples
    \item Goal is to calculate the total biomass of each taxon across the upper midwest
    \item Many grid cells have zero observed biomass
\end{itemize}

\note{The biomass is observed at corner points within each grid cell. We'd like to estimate the underlying distribution of biomass. Observed biomass was exactly zero in some grid cells, and the smoothed model should accomodate that.}
\end{frame}





\section{Motivation}


\begin{frame}{Motivation}
\subt{A look at the data}

\begin{center}
  \ig[width=\textwidth]{../../figures/prelim-talk/raw-biomass}
\end{center}

\note{The biomass is observed at corner points within each grid cell. We'd like to estimate the underlying distribution of biomass. Observed biomass was exactly zero in some grid cells, and the smoothed model should accomodate that.}
\end{frame}





\begin{frame}{Motivation}
\subt{A look at the data}

The data is non-negative with a long right tail, suggesting a log transformation or a gamma model.

\begin{center}
  \ig[width=0.55\textwidth]{../../figures/prelim-talk/raw-biomass-histogram}
\end{center}

\note{The biomass is observed at corner points within each grid cell. We'd like to estimate the underlying distribution of biomass. Observed biomass was exactly zero in some grid cells, and the smoothed model should accomodate that.}
\end{frame}






\section{Our approach}

\begin{frame}{Background}
\subt{Recall our approach}

Want a model for which power-law variation (long right tail) and exact zeroes are handled naturally, not as an exception.

\begin{itemize}
    \item We use the Tweedie family of distribution
    \item Tweedie distribution has a tuning parameter that slides smoothly from Poisson to Gamma distribution
    \item Estimate that tuning parameter via MLE
\end{itemize}

\note{The biomass is observed at corner points within each grid cell. We'd like to estimate the underlying distribution of biomass. Observed biomass was exactly zero in some grid cells, and the smoothed model should accomodate that.}
\end{frame}



\begin{comment}

\begin{frame}{Motivation}
\subt{Recall our approach}

Use a Generalized Additive Model (GAM) to smooth the biomass data

\begin{itemize}
    \item 
\end{itemize}

\note{The biomass is observed at corner points within each grid cell. We'd like to estimate the underlying distribution of biomass. Observed biomass was exactly zero in some grid cells, and the smoothed model should accomodate that.}
\end{frame}




\section{Simulation}

\begin{frame}{Simulation}
\subt{Setup}

Run 200 repetitions of this simulation:
\begin{itemize}
  \item $X_i \sim $N($0,1$), $i=1,\dots,100$
  \item $\beta = 1$
  \item $\nu = X \beta$
  \item $y \sim $Tweedie($\mu=\exp\left\{\nu\right\}, \theta=1.5$)
  \item Compare $\hat{\theta}$ from two methods:
  \begin{itemize}
    \item MLE
    \item Mean-variance matching
  \end{itemize}
\end{itemize}

\note{The biomass is observed at corner points within each grid cell. We'd like to estimate the underlying distribution of biomass. Observed biomass was exactly zero in some grid cells, and the smoothed model should accomodate that.}
\end{frame}



\begin{frame}{Simulation}
\subt{Response data}

\begin{center}
\ig[width=0.6\textwidth]{../../figures/Nov-conference-call/sim-y}
\end{center}

\note{The biomass is observed at corner points within each grid cell. We'd like to estimate the underlying distribution of biomass. Observed biomass was exactly zero in some grid cells, and the smoothed model should accomodate that.}
\end{frame}


\begin{frame}{Simulation}
\subt{Results for MLE}

\begin{center}
\ig[width=0.6\textwidth]{../../figures/Nov-conference-call/sim-mle}
\end{center}

\note{The biomass is observed at corner points within each grid cell. We'd like to estimate the underlying distribution of biomass. Observed biomass was exactly zero in some grid cells, and the smoothed model should accomodate that.}
\end{frame}



\begin{frame}{Simulation}
\subt{Results for mean-variance matching}

\begin{center}
\ig[width=0.6\textwidth]{../../figures/Nov-conference-call/sim-mean-var}
\end{center}

\note{The biomass is observed at corner points within each grid cell. We'd like to estimate the underlying distribution of biomass. Observed biomass was exactly zero in some grid cells, and the smoothed model should accomodate that.}
\end{frame}
\end{comment}



\section{Biomass data}

\begin{frame}{Biomass data}
\subt{Drawing biomass samples}

\begin{itemize}
  \item Create a model for each taxon
  \item Use MLE matching to estimate $\hat{\theta}$
  \item Use parametric bootstrap to get draws from biomass
  \item distribute draws from the ``total" taxon based on the relative biomass of the other taxa
\end{itemize}

\note{The biomass is observed at corner points within each grid cell. We'd like to estimate the underlying distribution of biomass. Observed biomass was exactly zero in some grid cells, and the smoothed model should accomodate that.}
\end{frame}


\begin{frame}{Biomass data}
\subt{Complications}

\begin{itemize}
  \item Smoothing biomass via a GAM for Wisconsin with 500 knots:
  \begin{itemize}
    \item Too few knots
    \item Takes too long to complete (some taxa never finished running)
  \end{itemize}
  \item Possible solution: adapt R-INLA for Tweedie likelihood
\end{itemize}

\note{The biomass is observed at corner points within each grid cell. We'd like to estimate the underlying distribution of biomass. Observed biomass was exactly zero in some grid cells, and the smoothed model should accomodate that.}
\end{frame}



\begin{frame}{Biomass data}
\subt{Complications}

\begin{itemize}
  \item Incomplete taxa:
  \begin{itemize}
    \item Cherries 
    \item Willow
    \item Walnuts
    \item Hickory
    \item Fir
    \item Spruce
    \item Cedar
    \item Hemlock
    \item Pine
    \item Birches
  \end{itemize}
\end{itemize}

\note{The biomass is observed at corner points within each grid cell. We'd like to estimate the underlying distribution of biomass. Observed biomass was exactly zero in some grid cells, and the smoothed model should accomodate that.}
\end{frame}




\begin{frame}{Biomass data}
\subt{Complications}

\begin{itemize}
  \item Complete taxa:
  \begin{itemize}
    \item tot
    \item Beech
    \item Ironwoods
    \item Basswood
    \item Ashes
    \item Elms
    \item Poplar
    \item Tamarack
    \item Maple
    \item Oaks
  \end{itemize}
\end{itemize}

\note{The biomass is observed at corner points within each grid cell. We'd like to estimate the underlying distribution of biomass. Observed biomass was exactly zero in some grid cells, and the smoothed model should accomodate that.}
\end{frame}


%Example slide:
%\begin{frame}{Title}
%\subt{Tweedie distribution}
%\bigskip
%
%\begin{itemize}
%  \item $f(y;\theta;\phi) = a(y,\phi) \exp \left[ \phi^{-1} \left\{ y\theta - %\kappa(\theta) \right\} \right]$
%  \item test:
%    \begin{align}
%          & \text{Since} & 2y &= 4x &  \notag\\
%          &              &  y &= x \notag
%    \end{align}
%\end{itemize}
%
%\note{note}
%\end{frame}

\begin{comment}

\begin{frame}{Title}
\subt{Tweedie distribution}
\bigskip

\begin{itemize}
  \item Exponential dispersion model:
    \begin{align}
        f(y;\theta;\phi) = a(y,\phi) \exp \left[ \phi^{-1} \left\{ y\theta - \kappa(\theta) \right\} \right] \notag
    \end{align}
  %\item test:
  %  \begin{align}
  %        & \text{Since} & 2y &= 4x &  \notag\\
  %        &              &  y &= x \notag
  %  \end{align}
\end{itemize}

\note{note}
\end{frame}



\begin{frame}{Title}
\subt{Tweedie distribution}
\bigskip

\begin{itemize}
  \item Tweedie model's parameters:
    \begin{align}
        \lambda &= \frac{\mu^(2-p)}{\phi (2-p)} \notag \\
        \alpha &= \frac{2-p}{1-p} \notag \\
        \gamma &= \phi(p-1)\mu^{p-1} \notag
    \end{align}
\end{itemize}

\note{note}
\end{frame}





\begin{frame}{Title}
\subt{Tweedie distribution}
\bigskip

\begin{itemize}
  \item Probability of exact zero:
    \begin{align}
        P(Y=0) = \exp \left\{ -\frac{\mu^{2-p}}{\phi (2-p)} \right\} \notag
    \end{align}
\end{itemize}

\note{note}
\end{frame}






%\begin{frame}{Title}
%\subt{Tweedie distribution}
%\bigskip
%
%
%
%\begin{itemize}
%  \item test:
%    \begin{align}
%        a(y,\phi) &= y^{-1} W(y,\phi, p) \notag \\
%        W_j &= \frac{y^{j\alpha}(p-1)^{j\alpha}}{\phi^{j(1-\alpha)}(2-p)^j j! %\Gamma(-j\alpha)} \notag
%    \end{align}
%\end{itemize}
%
%\note{note}
%\end{frame}


\end{comment}




\end{document}
